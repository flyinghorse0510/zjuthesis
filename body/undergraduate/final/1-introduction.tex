\cleardoublepage

\section{绪论}

\subsection{背景}

近年来,虚拟现实(VR)和增强现实(AR)技术取得了快速发展,涌现出许多令人兴奋的新技术。其中,手部和手指追踪技术在过去十年间已成为虚拟现实领域的热门研究课题。
手部和手指追踪技术的进展为各个领域的应用带来了新的可能性,包括游戏、培训模拟、医疗模拟、设计和建模以及人机交互等~\cite{introArticle1}。研究人员和开发者一直在积极探索不同的方法,如基于标记的跟踪、基于深度感知的方法、基于摄像头的方法和基于传感器的方法,以实现精确可靠的手部和手指追踪。

对于以定量分析为导向的手指运动追踪系统而言,实时性、稳定性、鲁棒性和准确性是其主要的要求和性能指标。只有满足这些要求,基于这项技术的各种应用才能充分实现其价值。由于手指运动追踪技术的理论意义和应用前景,它已经引起了国内外研究人员的广泛关注~\cite{introArticle4}。

在本章中,我们旨在全面概述手指运动追踪中的核心问题,包括追踪技术的实现原理、追踪过程中的整体性能、追踪技术的实施成本以及应用前景。我们将回顾在这些方面具有代表性的研究进展和方法,并概述手指运动追踪系统的交互应用。此外,我们还将讨论手指运动追踪技术的未来发展方向。

\subsection{国内外研究现状}

从原理上进行分析,目前的手指运动追踪系统可以大致分为两大类:
\begin{enumerate}
    \item {\bfseries 基于视觉的追踪系统}
    \item {\bfseries 基于传感器的追踪系统}
\end{enumerate}

基于视觉和基于传感器的追踪系统各有其优势和局限性:
\subsubsection{基于视觉的系统}

这些系统通常利用摄像头和计算机视觉算法来定位并追踪手指的运动。
通过捕捉手部的图像或视频,这些系统分析手指的视觉特征,比如手指轮廓或预先标记的图案,来估计手部和手指的姿态及相对位置~\cite{introArticle2}。

基于视觉的追踪系统通常需要良好的光照条件,并且可能对遮挡或手部外观的变化非常敏感。然而,它们具有非侵入性的优点,并且可以为手势识别等应用提供丰富的视觉信息。
\subsubsection{基于传感器的系统}

这些系统利用加速度计、陀螺仪和磁力计等各种传感器直接测量手指的运动和姿态。基于传感器的追踪系统可以捕捉手部和手指的物理运动,即使在复杂环境中也能实现较为准确和稳健的追踪~\cite{introArticle3}。这些系统可以克服在基于视觉的系统中存在的光照条件或遮挡的限制。
然而,此类系统可能经常需要进行校准并正确安装传感器以确保追踪结果的准确性。

\subsection{本研究的目的和意义}
当前,对于手指运动追踪系统而言,基于视觉的系统的主要优点是跟踪精度高,但存在对光强波动的敏感性和遮挡问题等缺点,可能导致较大的跟踪误差。此外,它也往往是笨重的和昂贵的~\cite{introArticle5}。

基于传感器的系统利用加速度计、陀螺仪、磁传感器、电阻传感器、光纤传感器等多种传感器直接检测手指的运动和弯曲,在一定程度上有助于缓解基于视觉的系统所存在的问题。此外,基于传感器的系统通常也更紧凑和便宜。然而,由于手指与身体的旋转和振动,加速度或惯性传感器通常容易受到重力的影响。另一方面,电阻性传感器的耐用度有限,因为它们在手指运动和弯曲的过程中受到机械应力的影响。此外,使用电阻传感器也很难检测到移动方向~\cite{introArticle6}。

与其他类型的传感器相比,由于磁传感器的多样性,其提供了测量更高精度的自由度的可能性。基于磁传感器的手指运动检测装置的信号源可以来自感应线圈、电磁铁、永磁铁甚至是地磁场,探测器可以是线圈、霍尔效应传感器或不同类型的磁电阻传感器。这些大量且不同的源/传感器组合允许为不同类型的应用开发具有最佳性能的设备。将磁传感器与加速度计、陀螺仪等其他类型的传感器进行融合,可以进一步提高检测系统的性能。

因此,本毕设项目开发了一种利用多个磁传感器和惯性传感器的,低成本、高灵敏度、紧凑的智能手指运动追踪系统,并同时具有实时手势识别功能。此外,深度学习技术也被积极探索和应用,以提高实时手势识别的准确性。
